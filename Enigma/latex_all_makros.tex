\newcommand{\script}{true}
%\newcommand{\script}{false} % beamer

\newcommand{\ausgefuellt}{true}
%\newcommand{\ausgefuellt}{false}

%\newcommand{\summary}{true}
\newcommand{\summary}{false}

%\newcommand{\deutsch}{true}
\newcommand{\deutsch}{false}
%\ifthenelse{\equal{\deutsch}{false}}
%{}
%{}

%\newcommand{\altedyn}{true}
\newcommand{\altedyn}{false}
%\ifthenelse{\equal{\altedyn}{false}}
%{}
%{}

\newcommand{\embedded}{false}
\newcommand{\comparch}{false}
\newcommand{\usaall}{true}




\usepackage{ifthen}

%rote Box (fuer Loesungen) ueber die ganze Seite mit Zeilenumbruch.
%\asxbox{blabla} 
\definecolor{hellrot}{rgb}{1,0.7,0.7}
\ifthenelse{\equal{\ausgefuellt}{true}}
{
\newcommand{\asxbox}[1]{
\begin{center}
\fcolorbox{black}{hellrot}
{\parbox{0.94\textwidth}
  {\textcolor{black}{#1}}
}
\end{center}
}
}
{
\newcommand{\asxbox}[1]{
\begin{center}
\fcolorbox{black}{hellrot}
{\parbox{0.94\textwidth}
  {\vspace{2cm}\hspace{3cm}}
}
\end{center}
}
}
%... fuer Folien
%\newcommand{\asxboxf}[1]{
%\begin{center}
%\fcolorbox{black}{hellrot}
%{\parbox{0.94\textwidth}
%  {\textcolor{hellrot}
%    {#1}
%  }
%}
%\end{center}
%}


%rote, kleine Box (fuer Loesungen) ohne Zeilenumbruch.
%\asybox{blabla}; \asyboxf fuer Folien
%\newcommand{\asybox}[1]{\colorbox{red}{\color{black}#1}}
%\newcommand{\asyboxf}[1]{\colorbox{red}{\color{red}#1}}
\ifthenelse{\equal{\ausgefuellt}{true}}
{
\newcommand{\asybox}[1]{\colorbox{red}{\color{black}#1}}
}
{
\newcommand{\asybox}[1]{\colorbox{red}{\color{red}#1}}
}

%gelbe Box um kleine, wichtige Dinge hevorzuheben
\newcommand{\asyboxy}[1]{\colorbox{yellow}{\color{black}#1}}

%hellgraue Box (fuer Bemerkungen, Erweiterungen) ueber die ganze Seite mit Zeilenumbruch.
\definecolor{hellgrau}{gray}{0.9}
\newcommand{\aszbox}[1]{
\begin{center}
\fcolorbox{black}{hellgrau}
{\parbox{0.95\textwidth}
  {\textcolor{black}
    {#1}
  }
}
\end{center}
}

%gelbe Box (fuer Formeln, Sï¿œtze) ueber die ganze Seite mit Zeilenumbruch.
\newcommand{\aszboxy}[1]{
\begin{center}
\fcolorbox{black}{yellow}
{\parbox{0.88\textwidth}
  {\textcolor{black}
    {#1}
  }
}
\end{center}
}

%weisse Box (fuer schmaelere Dinge) ueber die ganze Seite mit Zeilenumbruch.
%Parameter 1 ist der Text; Parameter 2 ist die Breite in % der Textbreite
\newcommand{\aswbox}[2]{
\begin{center}
\fcolorbox{black}{white}
{\parbox{#2\textwidth}
  {\textcolor{black}
    {#1}
  }
}
\end{center}
}

\newcommand{\asgboxy}[2]{
\begin{center}
\fcolorbox{black}{yellow}
{\parbox{#2\textwidth}
  {\textcolor{black}
    {#1}
  }
}
\end{center}
}

\newcommand{\asgbox}[2]{
\begin{center}
\fcolorbox{black}{hellgrau}
{\parbox{#2\textwidth}
  {\textcolor{black}
    {#1}
  }
}
\end{center}
}

%Regel:             \aszboxy{xx}      gelb
%Aufgabenlösung:    \asxbox{xx}       rot
%abgeleitete Regel: \aswbox{xxx}{0.8} weiß
%Prozeduren:        \aszbox{xx}       grau

\newcommand{\asr}[1]{\textcolor{red}{#1}}

%%%%%%%%%%%%%%%%%%%%%%%%%%%%%%%%%%%%%%%%%%%%%%%%%%%%%%%%%%%
% Babel
%%%%%%%%%%%%%%%%%%%%%%%%%%%%%%%%%%%%%%%%%%%%%%%%%%%%%%%%%%%
\newcommand{\aspara}[2]{
\begin{Parallel}{0.45\textwidth}{0.45\textwidth}
\ParallelLText{
\selectlanguage{english}
#1
}
\ParallelRText{
\selectlanguage{ngerman}
#2
}
\end{Parallel}
}

%%%%%%%%%%%%%%%%%%%%%%%%%%%%%%%%%%%%%%%%%%%%%%%%%%%%%%%%%%%
% Folien
%%%%%%%%%%%%%%%%%%%%%%%%%%%%%%%%%%%%%%%%%%%%%%%%%%%%%%%%%%%
\newcommand{\pause}{}

\newcommand{\asfolie}[3]{
\begin{frame}
\frametitle{#1}
\begin{block}{#2}
{#3}
\end{block}
\end{frame}
}

\newcommand{\asfolietheorem}[2]{
\begin{frame}
\frametitle{#1}
\begin{theorem}
{#2}
\end{theorem}
\end{frame}
}

\newcommand{\asfolieexample}[2]{
\begin{frame}
\frametitle{#1}
\begin{example}
{#2}
\end{example}
\end{frame}
}
%%%%%%%%%%%%%%%%%%%%%%%%%%%%%%%%%%%%%%%%%%%%%%%%%%%%%%%%%%%
% Modulhandbuch
%%%%%%%%%%%%%%%%%%%%%%%%%%%%%%%%%%%%%%%%%%%%%%%%%%%%%%%%%%%
% Inhalt / Content
\newcommand{\asmhcontd}[9]{
\begin{longtable}{| p{0.3\textwidth} | p{0.7\textwidth} |}
\caption{#1}\label{tab:#1}\\
%\centering
%\begin{tabularx}{\textwidth}{|p{4 cm}|X|}
  \hline
  \textbf{LSF-Reiter} & \textbf{Deutsch} \\
  \hline
  \textbf{Modul:} & #1 \\
  \hline
  \textbf{Zugehörige Veranstaltungen:} & #2 \\
  \hline
  \textbf{Inhalt:} & #3 \\
  \hline
  \textbf{Literatur:} & #4 \\
  \hline
  \textbf{Lehr- und Lernformen:} & #5 \\
  \hline
  \textbf{Studien-/ Prüfungsleistungen:} & #6 \\
  \hline
  \textbf{Geschätzter Workload:} & #7 \\
  \hline
  \textbf{Lehr- und Prüfungssprache:} & #8 \\
  \hline
  \textbf{Voraussetzungen für die Teilnahme:} & #9 \\
  \hline
%\end{tabularx}
\end{longtable}
}

\newcommand{\asmhconte}[9]{
\begin{longtable}{| p{0.3\textwidth} | p{0.7\textwidth} |}
\caption{#1}\label{tab:#1}\\
%\centering
%\begin{tabularx}{\textwidth}{|p{4 cm}|X|}
  \hline
  \textbf{LSF-Bar} & \textbf{English} \\
  \hline
  \textbf{Module:} & #1 \\
  \hline
  \textbf{Appendant lectures:} & #2 \\
  \hline
  \textbf{Content:} & #3 \\
  \hline
  \textbf{Literature:} & #4 \\
  \hline
  \textbf{Form of Lesson and Learning:} & #5 \\
  \hline
  \textbf{Premises for Allocation of Point of Performance:} & #6 \\
  \hline
  \textbf{Work:} & #7 \\
  \hline
  \textbf{Language of Exam and Lesson:} & #8 \\
  \hline
  \textbf{Premises for Participation:} & #9 \\
  \hline
%\end{tabularx}
\end{longtable}
}

% Learning Outcomes
% Wissen: Deutsch; alles und gesplittet
\newcommand{\asMhOutcDa}[3]{
\begin{longtable}{| p{0.2\textwidth} | p{0.35\textwidth} | p{0.15\textwidth}  | p{0.15\textwidth} |  p{0.15\textwidth} |}
\caption{#1}\label{tab:#1}\\
  \hline
  \textbf{Statement} & \textbf{Freitextfeld} & \textbf{Niveaustufe} & \textbf{Kompetenz} & \textbf{Kategorie} \\
  \hline
  \textbf{Die Studierenden haben ihr Wissen auf folgenden Gebieten erweitert und können dieses Wissen auch wiedergeben:} & #2 & wissen & Wissens-verbreiterung & Wissen und Verstehen\\
  \hline
  \textbf{Die Studierenden haben ihr bereits vorhandenes Wissen in folgenden Gebieten vertieft und können die entsprechenden Fachinhalte nicht nur wiedergeben, sondern auch erklären. Sie verstehen die Hintergründe, das Warum und Weshalb:} & #3 & verstehen & Wissens-vertiefung & Wissen und Verstehen\\
  \hline
\end{longtable}
}

\newcommand{\asMhOutcDaa}[1]{
\begin{longtable}{| p{0.2\textwidth} | p{0.35\textwidth} | p{0.15\textwidth}  | p{0.15\textwidth} |  p{0.15\textwidth} |}
\caption{#1}\label{tab:a#1}\\
  \hline
  \textbf{Statement} & \textbf{Freitextfeld} & \textbf{Niveaustufe} & \textbf{Kompetenz} & \textbf{Kategorie} \\
  \hline
\end{longtable}
}

\newcommand{\asMhOutcDab}[2]{
\begin{longtable}{| p{0.2\textwidth} | p{0.35\textwidth} | p{0.15\textwidth}  | p{0.15\textwidth} |  p{0.15\textwidth} |}
\caption{#1}\label{tab:b#1}\\
  \hline
  \textbf{Die Studierenden haben ihr Wissen auf folgenden Gebieten erweitert und können dieses Wissen auch wiedergeben:} & #2 & wissen & Wissens-verbreiterung & Wissen und Verstehen\\
  \hline
\end{longtable}
}

\newcommand{\asMhOutcDac}[2]{
\begin{longtable}{| p{0.2\textwidth} | p{0.35\textwidth} | p{0.15\textwidth}  | p{0.15\textwidth} |  p{0.15\textwidth} |}
\caption{#1}\label{tab:c#1}\\
  \hline
  \textbf{Die Studierenden haben ihr bereits vorhandenes Wissen in folgenden Gebieten vertieft und können die entsprechenden Fachinhalte nicht nur wiedergeben, sondern auch erklären. Sie verstehen die Hintergründe, das Warum und Weshalb:} & #2 & verstehen & Wissens-vertiefung & Wissen und Verstehen\\
  \hline
\end{longtable}
}


% Wissen: Englisch; alles und gesplittet
\newcommand{\asMhOutcEa}[3]{
\begin{longtable}{| p{0.2\textwidth} | p{0.35\textwidth} | p{0.15\textwidth}  | p{0.15\textwidth} |  p{0.15\textwidth} |}
\caption{#1}\label{tab:#1}\\
  \hline
  \textbf{Statement} & \textbf{Free text field} & \textbf{Level} & \textbf{Competence} & \textbf{Category} \\
  \hline
  \textbf{The students have broadened their knowledge in the following fields and are capable of reproducing this knowledge:} & #2 & realize & extending knowledge & knowledge and understanding\\
  \hline
  \textbf{The students have deepened their existing knowledge in the following areas and are capable of not only reproducing the corresponding contents but also of explaining them. They understand the underlying principles, the whys and wherefores.} & #3 & understand & consolidating knowledge & knowledge and understanding\\
  \hline
\end{longtable}
}

\newcommand{\asMhOutcEaa}[1]{
\begin{longtable}{| p{0.2\textwidth} | p{0.35\textwidth} | p{0.15\textwidth}  | p{0.15\textwidth} |  p{0.15\textwidth} |}
\caption{#1}\label{tab:a#1}\\
  \hline
  \textbf{Statement} & \textbf{Free text field} & \textbf{Level} & \textbf{Competence} & \textbf{Category} \\
  \hline
\end{longtable}
}

\newcommand{\asMhOutcEab}[2]{
\begin{longtable}{| p{0.2\textwidth} | p{0.35\textwidth} | p{0.15\textwidth}  | p{0.15\textwidth} |  p{0.15\textwidth} |}
\caption{#1}\label{tab:b#1}\\
  \hline
  \textbf{The students have broadened their knowledge in the following fields and are capable of reproducing this knowledge:} & #2 & realize & extending knowledge & knowledge and understanding\\
  \hline
\end{longtable}
}

\newcommand{\asMhOutcEac}[2]{
\begin{longtable}{| p{0.2\textwidth} | p{0.35\textwidth} | p{0.15\textwidth}  | p{0.15\textwidth} |  p{0.15\textwidth} |}
\caption{#1}\label{tab:c#1}\\
  \hline
  \textbf{The students have deepened their existing knowledge in the following areas and are capable of not only reproducing the corresponding contents but also of explaining them. They understand the underlying principles, the whys and wherefores.} & #2 & understand & consolidating knowledge & knowledge and understanding\\
  \hline
\end{longtable}
}

% Instrumental: Deutsch; alles und gesplittet
\newcommand{\asMhOutcDb}[5]{
\begin{longtable}{| p{0.2\textwidth} | p{0.35\textwidth} | p{0.15\textwidth}  | p{0.15\textwidth} |  p{0.15\textwidth} |}
  \hline
  \textbf{Die Studierenden können das Wissen aus folgenden Themenbereichen praktisch anwenden:} & #1 & anwenden & Instrumentale Kompetenz & Können\\
  \hline
  \textbf{Die Studierenden können ihr Wissen aus folgenden Themenbereichen nicht nur praktisch anwenden, sie können darüber hinaus auch ihr Vorgehen beim Theorie-Praxis-Transfer und dessen Ergebnis beurteilen:} & #2 & evaluieren / beurteilen & Instrumentale Kompetenz & Können\\
  \hline
  \textbf{Die Studierenden können ihr Wissen nicht nur anwenden und das Anwendungsverfahren und / oder Anwendungsergebnis beurteilen, sie können darüber hinaus auch eigenständig weiterführende Fragestellungen in folgenden Bereichen entwickeln:} & #3 & erschaffen & Instrumentale Kompetenz & Können\\
  \hline
  \textbf{Die Studierenden haben durch die Belegung des Moduls auf folgende Art und Weise ihre Fähigkeit verbessert und ihre Bereitschaft erhöht, Informationen aufzunehmen und bei der Lösung von Problemen zu berücksichtigen:} & #4 & empfangen & Instrumentale Kompetenz & Können\\
  \hline
  \textbf{Die Studierenden haben ihre Fähigkeit und Bereitschaft zur aktiven Teilnahme am eigenen Lernen auf folgender Art und Weise erhöht:} & #5 & reagieren & Instrumentale Kompetenz & Können\\
  \hline
\end{longtable}
}

\newcommand{\asMhOutcDba}[1]{
\begin{longtable}{| p{0.2\textwidth} | p{0.35\textwidth} | p{0.15\textwidth}  | p{0.15\textwidth} |  p{0.15\textwidth} |}
  \hline
  \textbf{Die Studierenden können das Wissen aus folgenden Themenbereichen praktisch anwenden:} & #1 & anwenden & Instrumentale Kompetenz & Können\\
  \hline
\end{longtable}
}

\newcommand{\asMhOutcDbb}[1]{
\begin{longtable}{| p{0.2\textwidth} | p{0.35\textwidth} | p{0.15\textwidth}  | p{0.15\textwidth} |  p{0.15\textwidth} |}
  \hline
  \textbf{Die Studierenden können ihr Wissen aus folgenden Themenbereichen nicht nur praktisch anwenden, sie können darüber hinaus auch ihr Vorgehen beim Theorie-Praxis-Transfer und dessen Ergebnis beurteilen:} & #1 & evaluieren / beurteilen & Instrumentale Kompetenz & Können\\
  \hline
\end{longtable}
}

\newcommand{\asMhOutcDbc}[1]{
\begin{longtable}{| p{0.2\textwidth} | p{0.35\textwidth} | p{0.15\textwidth}  | p{0.15\textwidth} |  p{0.15\textwidth} |}
  \hline
  \textbf{Die Studierenden können ihr Wissen nicht nur anwenden und das Anwendungsverfahren und / oder Anwendungsergebnis beurteilen, sie können darüber hinaus auch eigenständig weiterführende Fragestellungen in folgenden Bereichen entwickeln:} & #1 & erschaffen & Instrumentale Kompetenz & Können\\
  \hline
\end{longtable}
}

\newcommand{\asMhOutcDbd}[1]{
\begin{longtable}{| p{0.2\textwidth} | p{0.35\textwidth} | p{0.15\textwidth}  | p{0.15\textwidth} |  p{0.15\textwidth} |}
  \hline
  \textbf{Die Studierenden haben durch die Belegung des Moduls auf folgende Art und Weise ihre Fähigkeit verbessert und ihre Bereitschaft erhöht, Informationen aufzunehmen und bei der Lösung von Problemen zu berücksichtigen:} & #1 & empfangen & Instrumentale Kompetenz & Können\\
  \hline
\end{longtable}
}

\newcommand{\asMhOutcDbe}[1]{
\begin{longtable}{| p{0.2\textwidth} | p{0.35\textwidth} | p{0.15\textwidth}  | p{0.15\textwidth} |  p{0.15\textwidth} |}
  \hline
  \textbf{Die Studierenden haben ihre Fähigkeit und Bereitschaft zur aktiven Teilnahme am eigenen Lernen auf folgender Art und Weise erhöht:} & #1 & reagieren & Instrumentale Kompetenz & Können\\
  \hline
\end{longtable}
}

% Instrumental: Englisch; alles und gesplittet
\newcommand{\asMhOutcEb}[5]{
\begin{longtable}{| p{0.2\textwidth} | p{0.35\textwidth} | p{0.15\textwidth}  | p{0.15\textwidth} |  p{0.15\textwidth} |}
  \hline
  \textbf{The students are capable of applying the knowledge they have acquired in the following fields:} & #1 & apply & instrumental competence & skills\\
  \hline
  \textbf{The students are capable of applying the knowledge they have acquired in the following fields and, additionally, of assessing their own approach to the theory-praxis-transfer and the result thereof:} & #2 & evaluate / beurteilen & instrumental competence & skills\\
  \hline
  \textbf{The students can not only apply their knowledge and assess the application methods and / or results, they can also independently develop further research questions in the following fields:} & #3 & create & instrumental competence & skills\\
  \hline
  \textbf{By enrolling in the module, the students have enhanced their ability and their readiness to take up information and consider it in their problem solution process in the following way:} & #4 & receive & instrumental competence & skills\\
  \hline
  \textbf{The students have enhanced their ability and their readiness to actively participate in their own learning process in the following way:} & #5 & react & instrumental competence & skills\\
  \hline
\end{longtable}
}

\newcommand{\asMhOutcEba}[1]{
\begin{longtable}{| p{0.2\textwidth} | p{0.35\textwidth} | p{0.15\textwidth}  | p{0.15\textwidth} |  p{0.15\textwidth} |}
  \hline
  \textbf{The students are capable of applying the knowledge they have acquired in the following fields:} & #1 & apply & instrumental competence & skills\\
  \hline
\end{longtable}
}

\newcommand{\asMhOutcEbb}[1]{
\begin{longtable}{| p{0.2\textwidth} | p{0.35\textwidth} | p{0.15\textwidth}  | p{0.15\textwidth} |  p{0.15\textwidth} |}
  \hline
  \textbf{The students are capable of applying the knowledge they have acquired in the following fields and, additionally, of assessing their own approach to the theory-praxis-transfer and the result thereof:} & #1 & evaluate / beurteilen & instrumental competence & skills\\
  \hline
\end{longtable}
}

\newcommand{\asMhOutcEbc}[1]{
\begin{longtable}{| p{0.2\textwidth} | p{0.35\textwidth} | p{0.15\textwidth}  | p{0.15\textwidth} |  p{0.15\textwidth} |}
  \hline
  \textbf{The students can not only apply their knowledge and assess the application methods and / or results, they can also independently develop further research questions in the following fields:} & #1 & create & instrumental competence & skills\\
  \hline
\end{longtable}
}

\newcommand{\asMhOutcEbd}[1]{
\begin{longtable}{| p{0.2\textwidth} | p{0.35\textwidth} | p{0.15\textwidth}  | p{0.15\textwidth} |  p{0.15\textwidth} |}
  \hline
  \textbf{By enrolling in the module, the students have enhanced their ability and their readiness to take up information and consider it in their problem solution process in the following way:} & #1 & receive & instrumental competence & skills\\
  \hline
\end{longtable}
}

\newcommand{\asMhOutcEbe}[1]{
\begin{longtable}{| p{0.2\textwidth} | p{0.35\textwidth} | p{0.15\textwidth}  | p{0.15\textwidth} |  p{0.15\textwidth} |}
  \hline
  \textbf{The students have enhanced their ability and their readiness to actively participate in their own learning process in the following way:} & #1 & react & instrumental competence & skills\\
  \hline
\end{longtable}
}





% Systemisch: Deutsch; alles und gesplittet
\newcommand{\asMhOutcDc}[6]{
\begin{longtable}{| p{0.2\textwidth} | p{0.35\textwidth} | p{0.15\textwidth}  | p{0.15\textwidth} |  p{0.15\textwidth} |}
  \hline
  \textbf{Die Studierenden können nicht nur mit einfachen sondern auch mit folgenden komplexen Sachverhalten umgehen und entsprechend handeln:} & #1 & anwenden & Systemische Kompetenz & Können\\
  \textbf{Die Studierenden haben im Laufe ihres Studiums bereits ein Wissens- und Verstehensniveau erreicht, das sie befähigt, nicht nur einfache sondern auch komplexere Zusammenhänge zu analysieren. Sie können darauf aufbauend wissenschaftliche oder praxisbezogene Fragestellungen in folgenden Fachgebieten selbständig identifizieren / entdecken:} & #2 & analysieren & Systemische Kompetenz & Können\\
  \textbf{Die Studierenden haben im Laufe ihres Studiums nicht nur ein Wissens- und Verstehensniveau erreicht, das sie befähigt, komplexere Zusammenhänge zu analysieren und darauf aufbauend wissenschaftliche oder praxisbezogene Fragestellungen selbständig zu identifizieren / zu entdecken. Sie können auch Problemlösungen für folgende komplexe Fragestellungen entwickeln und so einen Beitrag für die Weiterentwicklung von Wissenschaft /Gesellschaft /Praxis leisten:} & #3 & erschaffen & Systemische Kompetenz & Können\\
  \textbf{Die Studierenden haben durch die Teilnahme an den Lehrveranstaltungen des Moduls im Wege der Beteiligung an demokratischen Prozessen oder durch die Übernahme sozialer Verantwortung die Bereitschaft erlangt, die folgenden gesellschaftliche Werte zu akzeptieren oder sich ihnen zu verpflichten:} & #4 & werten & Systemische Kompetenz & Können\\
  \textbf{Die Studierenden haben durch die Teilnahme an den Lehrveranstaltungen des Moduls in folgenden Themenfeldern die Bereitschaft entwickelt, unterschiedliche gesellschaftsbezogene Wertvorstellungen oder divergierende professionelle ethische Standards problembezogen abzuwägen. Alternativ haben die Studierenden gelernt, Wertvorstellungen, die von den eigenen abweichen, zu akzeptieren.} & #5 & organisieren von Werten & Systemische Kompetenz & Können\\
  \textbf{Die Studierenden haben durch die Teilnahme an den Lehrveranstaltungen des Moduls ihre eigenen Wertvorstellungen und Wertpräferenzen in Bezug auf folgende Themenbereiche geklärt:} & #6 & charakterisieren von Werten & Systemische Kompetenz & Können\\
  \hline
\end{longtable}
}

\newcommand{\asMhOutcDca}[1]{
\begin{longtable}{| p{0.2\textwidth} | p{0.35\textwidth} | p{0.15\textwidth}  | p{0.15\textwidth} |  p{0.15\textwidth} |}
  \hline
  \textbf{Die Studierenden können nicht nur mit einfachen sondern auch mit folgenden komplexen Sachverhalten umgehen und entsprechend handeln:} & #1 & anwenden & Systemische Kompetenz & Können\\
  \hline
\end{longtable}
}

\newcommand{\asMhOutcDcb}[1]{
\begin{longtable}{| p{0.2\textwidth} | p{0.35\textwidth} | p{0.15\textwidth}  | p{0.15\textwidth} |  p{0.15\textwidth} |}
  \hline
  \textbf{Die Studierenden haben im Laufe ihres Studiums bereits ein Wissens- und Verstehensniveau erreicht, das sie befähigt, nicht nur einfache sondern auch komplexere Zusammenhänge zu analysieren. Sie können darauf aufbauend wissenschaftliche oder praxisbezogene Fragestellungen in folgenden Fachgebieten selbständig identifizieren / entdecken:} & #1 & analysieren & Systemische Kompetenz & Können\\
  \hline
\end{longtable}
}

\newcommand{\asMhOutcDcc}[1]{
\begin{longtable}{| p{0.2\textwidth} | p{0.35\textwidth} | p{0.15\textwidth}  | p{0.15\textwidth} |  p{0.15\textwidth} |}
  \hline
  \textbf{Die Studierenden haben im Laufe ihres Studiums nicht nur ein Wissens- und Verstehensniveau erreicht, das sie befähigt, komplexere Zusammenhänge zu analysieren und darauf aufbauend wissenschaftliche oder praxisbezogene Fragestellungen selbständig zu identifizieren / zu entdecken. Sie können auch Problemlösungen für folgende komplexe Fragestellungen entwickeln und so einen Beitrag für die Weiterentwicklung von Wissenschaft /Gesellschaft /Praxis leisten:} & #1 & erschaffen & Systemische Kompetenz & Können\\
  \hline
\end{longtable}
}

\newcommand{\asMhOutcDcd}[1]{
\begin{longtable}{| p{0.2\textwidth} | p{0.35\textwidth} | p{0.15\textwidth}  | p{0.15\textwidth} |  p{0.15\textwidth} |}
  \hline
  \textbf{Die Studierenden haben durch die Teilnahme an den Lehrveranstaltungen des Moduls im Wege der Beteiligung an demokratischen Prozessen oder durch die Übernahme sozialer Verantwortung die Bereitschaft erlangt, die folgenden gesellschaftliche Werte zu akzeptieren oder sich ihnen zu verpflichten:} & #1 & werten & Systemische Kompetenz & Können\\
  \hline
\end{longtable}
}

\newcommand{\asMhOutcDce}[1]{
\begin{longtable}{| p{0.2\textwidth} | p{0.35\textwidth} | p{0.15\textwidth}  | p{0.15\textwidth} |  p{0.15\textwidth} |}
  \hline
  \textbf{Die Studierenden haben durch die Teilnahme an den Lehrveranstaltungen des Moduls in folgenden Themenfeldern die Bereitschaft entwickelt, unterschiedliche gesellschaftsbezogene Wertvorstellungen oder divergierende professionelle ethische Standards problembezogen abzuwägen. Alternativ haben die Studierenden gelernt, Wertvorstellungen, die von den eigenen abweichen, zu akzeptieren.} & #1 & organisieren von Werten & Systemische Kompetenz & Können\\
  \hline
\end{longtable}
}

\newcommand{\asMhOutcDcf}[1]{
\begin{longtable}{| p{0.2\textwidth} | p{0.35\textwidth} | p{0.15\textwidth}  | p{0.15\textwidth} |  p{0.15\textwidth} |}
  \hline
  \textbf{Die Studierenden haben durch die Teilnahme an den Lehrveranstaltungen des Moduls ihre eigenen Wertvorstellungen und Wertpräferenzen in Bezug auf folgende Themenbereiche geklärt:} & #1 & charakterisieren von Werten & Systemische Kompetenz & Können\\
  \hline
\end{longtable}
}

% Systemisch: Englisch; alles und gesplittet
\newcommand{\asMhOutcEc}[6]{
\begin{longtable}{| p{0.2\textwidth} | p{0.35\textwidth} | p{0.15\textwidth}  | p{0.15\textwidth} |  p{0.15\textwidth} |}
  \hline
  \textbf{The students can not only handle simple but also the following complex issues and act accordingly:} & #1 & apply & systemic competence & skills\\
  \textbf{In the course of their study, the students have already reached a level of knowledge and understanding that enables them to analyze not only simple but also complex interactions. On this basis, they are capable of independently identifying scientific or practice-related issues in the following fields:} & #2 & analyze & systemic competence & skills\\
  \textbf{In the course of their study, the students have already reached a level of knowledge and understanding that enables them to analyze not only simple but also complex interactions. On this basis, they are capable of independently identifying scientific or practice-related issues. They can also develop solutions to problems for the following complex issues and thus make a contribution to the further development of science / society / practice:} & #3 & create & systemic competence & skills\\
  \textbf{By attending the courses of this module, by participating in democratic processes or taking on social responsibility, the students have developed the willingness to accept or commit themselves to the following social values:} & #4 & grade & systemic competence & skills\\
  \textbf{By attending the courses of this module, the students have developed the willingness to weigh the problems pertaining to different social values or diverging ethical standards of their profession. Alternatively, the students have learned to accept values that differ from their own.} & #5 & organize values & systemic competence & skills\\
  \textbf{By attending the courses of this module, the students have become clear about their own values and value-related preferences with regard to the following fields:} & #6 & characterize values & systemic competence & skills\\
  \hline
\end{longtable}
}

\newcommand{\asMhOutcEca}[1]{
\begin{longtable}{| p{0.2\textwidth} | p{0.35\textwidth} | p{0.15\textwidth}  | p{0.15\textwidth} |  p{0.15\textwidth} |}
  \hline
  \textbf{The students can not only handle simple but also the following complex issues and act accordingly:} & #1 & apply & systemic competence & skills\\
  \hline
\end{longtable}
}

\newcommand{\asMhOutcEcb}[1]{
\begin{longtable}{| p{0.2\textwidth} | p{0.35\textwidth} | p{0.15\textwidth}  | p{0.15\textwidth} |  p{0.15\textwidth} |}
  \hline
  \textbf{In the course of their study, the students have already reached a level of knowledge and understanding that enables them to analyze not only simple but also complex interactions. On this basis, they are capable of independently identifying scientific or practice-related issues in the following fields:} & #1 & analyze & systemic competence & skills\\
  \hline
\end{longtable}
}

\newcommand{\asMhOutcEcc}[1]{
\begin{longtable}{| p{0.2\textwidth} | p{0.35\textwidth} | p{0.15\textwidth}  | p{0.15\textwidth} |  p{0.15\textwidth} |}
  \hline
  \textbf{In the course of their study, the students have already reached a level of knowledge and understanding that enables them to analyze not only simple but also complex interactions. On this basis, they are capable of independently identifying scientific or practice-related issues. They can also develop solutions to problems for the following complex issues and thus make a contribution to the further development of science / society / practice:} & #1 & create & systemic competence & skills\\
  \hline
\end{longtable}
}

\newcommand{\asMhOutcEcd}[1]{
\begin{longtable}{| p{0.2\textwidth} | p{0.35\textwidth} | p{0.15\textwidth}  | p{0.15\textwidth} |  p{0.15\textwidth} |}
  \hline
  \textbf{By attending the courses of this module, by participating in democratic processes or taking on social responsibility, the students have developed the willingness to accept or commit themselves to the following social values:} & #1 & grade & systemic competence & skills\\
  \hline
\end{longtable}
}

\newcommand{\asMhOutcEce}[1]{
\begin{longtable}{| p{0.2\textwidth} | p{0.35\textwidth} | p{0.15\textwidth}  | p{0.15\textwidth} |  p{0.15\textwidth} |}
  \hline
  \textbf{By attending the courses of this module, the students have developed the willingness to weigh the problems pertaining to different social values or diverging ethical standards of their profession. Alternatively, the students have learned to accept values that differ from their own.} & #1 & organize values & systemic competence & skills\\
  \hline
\end{longtable}
}

\newcommand{\asMhOutcEcf}[1]{
\begin{longtable}{| p{0.2\textwidth} | p{0.35\textwidth} | p{0.15\textwidth}  | p{0.15\textwidth} |  p{0.15\textwidth} |}
  \hline
  \textbf{By attending the courses of this module, the students have become clear about their own values and value-related preferences with regard to the following fields:} & #1 & characterize values & systemic competence & skills\\
  \hline
\end{longtable}
}


% Kommunikativ: Deutsch; alles und gesplittet
\newcommand{\asMhOutcDd}[2]{
\begin{longtable}{| p{0.2\textwidth} | p{0.35\textwidth} | p{0.15\textwidth}  | p{0.15\textwidth} |  p{0.15\textwidth} |}
  \hline
  \textbf{Die Studierenden können sich sprachlich effektiv austauschen. Sie haben durch die Belegung des Moduls ihre Kommunikationsfähigkeiten in folgenden Bereichen (fachlich/ allgemein/Fremdsprache) verbessert:} & #1 & anwenden & Kommunikative Kompetenz & Können\\
  \textbf{Die Studierenden können in der Diskussion über folgende Themen ihre Meinung begründet darlegen und abweichende Meinungen akzeptieren:} & #2 & werten & Kommunikative Kompetenz & Können\\
  \hline
\end{longtable}
}

% Kommunikativ: Englisch; alles und gesplittet
\newcommand{\asMhOutcEd}[2]{
\begin{longtable}{| p{0.2\textwidth} | p{0.35\textwidth} | p{0.15\textwidth}  | p{0.15\textwidth} |  p{0.15\textwidth} |}
  \hline
  \textbf{The students are capable of communicating effectively. By attending the module, they have improved their communicative skills in the following fields (technical / general / foreign language):} & #1 & apply & communicative competence & skills\\
  \textbf{In discussions, the students are able to make their own opinion clear in a well-argued way and to accept dissenting views in the following fields:} & #2 & grade & communicative competence & skills \\
  \hline
\end{longtable}
}


\newcommand{\ModuleNameD}{Elektrotechnik/Physik 2: Elektrodynamik}
\newcommand{\ModuleNameE}{Electrical Engineering/Physics 2: Electrodynamics}
\newcommand{\LectureaNameD}{Elektrodynamik}
\newcommand{\LectureaNameE}{Electrodynamics}
\newcommand{\LecturebNameD}{b}
\newcommand{\LecturebNameE}{b}
\newcommand{\LecturecNameD}{c}
\newcommand{\LecturecNameE}{c}
\newcommand{\LecturedNameD}{d}
\newcommand{\LecturedNameE}{d}

%%%%%%%%%%%%%%%%%%%%%%%%%%%%%%%%%%%%%%%%%%%%%%%%%%%%%%%%%%%
% Spezifkation
%%%%%%%%%%%%%%%%%%%%%%%%%%%%%%%%%%%%%%%%%%%%%%%%%%%%%%%%%%%

\newcommand{\chipname}{\textbf{HS-Weingarten Phase ASIC} }
\newcommand{\designUnita}{\textbf{ccreg} }
\newcommand{\designUnitb}{\textbf{ucreg} }
\newcommand{\ioa}{\textbf{S0\_i} }
\newcommand{\iob}{\textbf{S1\_i} }
\newcommand{\ioc}{\textbf{SP\_i} }
\newcommand{\iod}{\textbf{SX\_i} }
\newcommand{\ioe}{\textbf{ena\_i} }
\newcommand{\iof}{\textbf{adr\_i} }
\newcommand{\iog}{\textbf{di\_i(3:0)} }
\newcommand{\insiga}{\textbf{rx\_ready\_s} }
\newcommand{\insigb}{\textbf{phase\_ready\_s} }
\newcommand{\supplya}{\textbf{VDD\_MAIN} }
\newcommand{\ChipClka}{\textbf{clk\_i} }
\newcommand{\ChipClkaspeed}{\textbf{100MHz} }
\newcommand{\ChipRsta}{\textbf{rst\_n\_i} }
\newcommand{\ChipIOa}{\textbf{tx\_o} }
\newcommand{\ChipIOb}{\textbf{rx\_i} }


\newcommand{\designUnitUart}{\textbf{UART}}
\newcommand{\UartClkSpeed}{\textbf{50MHz} }
\newcommand{\UartClka}{\textbf{clk\_i}}
\newcommand{\UartRsta}{\textbf{rst\_n\_i}}
\newcommand{\UartIOa}{\textbf{tx\_o}}
\newcommand{\UartIOb}{\textbf{rx\_i}}
\newcommand{\UartIOc}{\textbf{aaaascii\_o}}
\newcommand{\UartIOd}{\textbf{aaaascii\_i}}
\newcommand{\UartIOe}{\textbf{rx\_ready\_o}}
\newcommand{\UartIOf}{\textbf{tx\_ready\_o}}
\newcommand{\UartIOg}{\textbf{txxxxx\_start\_i}}
\newcommand{\UartIOh}{\textbf{reg\_sta\_in\_i(bw:0)}}
\newcommand{\UartIOi}{\textbf{reg\_sta\_out\_o(bw:0)}}
\newcommand{\UartIOj}{\textbf{reg\_sta\_en\_i}}
\newcommand{\UartIOk}{\textbf{reg\_cnf\_in\_i(bw:0)}}
\newcommand{\UartIOl}{\textbf{reg\_cnf\_out\_o(bw:0)}}
\newcommand{\UartIOm}{\textbf{reg\_cnf\_en\_i}}
\newcommand{\UartIOn}{\textbf{reg\_brt\_in\_i(bw:0)}}
\newcommand{\UartIOo}{\textbf{reg\_brt\_out\_o(bw:0)}}
\newcommand{\UartIOp}{\textbf{reg\_brt\_en\_i}}
\newcommand{\UartIOq}{\textbf{reg\_tx\_in\_i(bw:0)}}
\newcommand{\UartIOr}{\textbf{reg\_tx\_out\_o(bw:0)}}
\newcommand{\UartIOs}{\textbf{reg\_tx\_en\_i}}
\newcommand{\UartIOt}{\textbf{reg\_rx\_in\_i(bw:0)}}
\newcommand{\UartIOu}{\textbf{reg\_rx\_out\_o(bw:0)}}
\newcommand{\UartIOv}{\textbf{reg\_rx\_en\_i}}

\newcommand{\UartIRQeins}{\textbf{RX\_ERR\_IRQ}}
\newcommand{\UartIRQzwei}{\textbf{RX\_RDY\_IRQ}}
\newcommand{\UartIRQdrei}{\textbf{TX\_RDY\_IRQ}}

\newcommand{\UartREGnulleins}{\textbf{UART01\_BPI\_CLC}}
\newcommand{\UartREGnullzwei}{\textbf{UART01\_BPI\_CNF}}
\newcommand{\UartREGnulldrei}{\textbf{UART01\_BPI\_ID}}

\newcommand{\UartREGeins}{\textbf{UART01\_KERN\_RX}}
\newcommand{\UartREGzwei}{\textbf{UART01\_KERN\_TX}}
\newcommand{\UartREGdrei}{\textbf{UART01\_KERN\_CNF}}
\newcommand{\UartREGvier}{\textbf{UART01\_KERN\_STA}}
\newcommand{\UartREGfuenf}{\textbf{UART01\_KERN\_BRT}}

\newcommand{\UartBITeins}{\textbf{start\_tx}}
\newcommand{\UartBITzwei}{\textbf{par\_err\_rx}}
\newcommand{\UartBITzweia}{\textbf{par\_rx}}
\newcommand{\UartBITdrei}{\textbf{rx\_ready}}
\newcommand{\UartBITvier}{\textbf{tx\_ready}}
\newcommand{\UartBITfuenf}{\textbf{baud\_rate}}
\newcommand{\UartBITsechs}{\textbf{stop\_bit}}
\newcommand{\UartBITsieben}{\textbf{parity}}
\newcommand{\UartBITacht}{\textbf{nr\_bits}}


\newcommand{\gpsSats}{$31$ }
\newcommand{\gpsOrb}{$20180 km$ }
\newcommand{\gpsHeight}{$20180 km$ }
\newcommand{\gpsTilt}{$55^\circ$ }
\newcommand{\gpsFreq}{$1575,42 MHz$ }



\newcommand{\asregkopf}[2]{
\begin{center}
\begin{tabularx}{\textwidth}{Xr}
  %{#1} & \begin{flushright}[Reset value: {#2}]\end{flushright}
  {#1} & [Reset value: {#2}]
\end{tabularx}
\end{center}
}

\newcommand{\asregtwoxsixteen}[4]{
\begin{center}
\begin{tabularx}{\textwidth}{|X|X|X|X|X|X|X|X|X|X|X|X|X|X|X|X|}
  \hline
  31 & 30 & 29 & 28 & 27 & 26 & 25 & 24 & 23 & 22 & 21 & 20 & 19 & 18 & 17 & 16  \\
  \hline
  #1 \\
  \hline
  #2 \\
  \hline
  \hline
  15 & 14 & 13 & 12 & 11 & 10 & 09 & 08 & 07 & 06 & 05 & 04 & 03 & 02 & 01 & 00 \\
  \hline
  #3 \\
  \hline
  #4 \\
  \hline
\end{tabularx}
\end{center}
}

\newcommand{\asregfour}[2]{
\begin{center}
\begin{tabularx}{\textwidth}{|X|X|X|X|}
  \hline
  3 & 2 & 1 & 0 \\
  \hline
  #1 \\
  \hline
  #2 \\
  \hline
\end{tabularx}
\end{center}
}

\newcommand{\asregeight}[2]{
\begin{center}
\begin{tabularx}{\textwidth}{|X|X|X|X|X|X|X|X|}
  \hline
  7 & 6 & 5 & 4 & 3 & 2 & 1 & 0 \\
  \hline
  #1 \\
  \hline
  #2 \\
  \hline
\end{tabularx}
\end{center}
}

\newcommand{\asregdesc}[1]{
\begin{center}
\begin{tabularx}{\textwidth}{|l|l|l|X|}
  \hline
  \textbf{Field} & \textbf{Bits} & \textbf{Type} & \textbf{Description} \\
  \hline
  #1 \\
  \hline
\end{tabularx}
\end{center}
}

\newcommand{\asket}[1]{\mid #1 \rangle}
\newcommand{\asbra}[1]{\langle #1 \mid}

%\setlength{\parindent}{0pt} 
%\setlength{\parskip}{5pt plus 2pt minus 1pt}

%\setlength{\parskip}{\baselineskip}
