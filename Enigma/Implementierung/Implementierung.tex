\chapter{Implementierung}\label{ch:impl}
%\section{Enigma-Maschine}\label{sec:impl_enigma}
%Um die Bombe in Software Nachzubilden, muss zuerst eine Enigma-Maschine nachgebildet werden.
%\subsection{Rotoren}\label{subsec:impl_enigma_rotor}
%Die Verdrahtung der Rotoren wurden als Vektor realisiert.
%Der zu permutierende Buchstabe wird hierfür als Index in den Vektor genutzt.




\section{Turing-Welchman-Bombe}\label{sec:impl_bombe}
%\section{Algorithmus Bombe}\label{sec:algorithmus-bombe}
%
%\begin{algorithm}[htbp]
%	\caption{Bombe Algorithmus}
%	\begin{algorithmic}
	%		\Procedure{Bombe}{$p_0 \dots p_{n-1} :$ [Char], $c_0 \dots c_{n-1} :$ [Char]}
	%		\ForAll{\textsl{rotors} $\in$ \textbf{permut(rotor order)}, \textsl{pos} $\in$ \textbf{[AAA $\dots$ ZZZ]}}
	%		\State plugs: Char $\rightarrow \{ $Char$\}$
	%		\State plugs$(p_0)\ \cup= \{\textbf{'A'}\}$
	%		\While{\textsl{plugs} changing}
	%		\ForAll{\textsl{i} $\in$ \textbf{[0\dots n--1]}}
	%		\State plugs($c_i$) $\cup=$ $\bigcup_{p \in \text{plugs}(p_i)}$ encrypt(\textsl{rotors, p, pos}+\textsl{i})
	%		\State plugs($p_i$) $\cup=$ $\bigcup_{p \in \text{plugs}(c_i)}$ encrypt(\textsl{rotors, p, pos}+\textsl{i})
	%		\EndFor
	%		\EndWhile
	%		\If{$\forall$ \textsl{S} $\in$ \textbf{cod(plugs)}: \#S < \#Char}
	%		\State report(\textsl{pos, plugs})
	%		\EndIf
	%		\EndFor
	%		\EndProcedure
	%	\end{algorithmic}
%	\label{alg:algorithm}1
%\end{algorithm}
%
\subsection{Menü Algorithmus}\label{subsec:cycle-finding-algorithm}
Um eine Turing-Welchman-Bombe in der Programmiersprache C nachzubilden, muss zuerst ein Algorithmus entworfen werden, welcher das Menü durch ein von dem Nutzer vorgegebenes Crib und 
%TODO
Geheimtext bildet.

Hierfür werden die Knoten als Struktur dargestellt, welche zum einen den Buchstaben und zum anderen einen Vektor mit den anliegenden Auslegern beinhaltet. 
Die Buchstaben-Tupel wurden ebenfalls als Struktur dargestellt, welche zwei Knoten, die Position im Crib und einen Booleschen Wert beinhaltet welcher aussagt, ob dieses Tupel zum Zyklus beiträgt.


%\section{Software Implementierung}\label{sec:implementierung_bombe}

\noindent
\lstinputlisting[style=mystyle, caption={Realisierung der Menü Strukturen}, label={code:impl_menu}]{Implementierung/menu_structs.c}
Die Tupel werden nun in einer \glqq Nachschlagetabelle\grqq{} abgelegt.
Diese Tabelle hat 26 Stellen, repräsentativ für das Alphabet.
Ein jeweiliges Tupel wird sowohl unter dem erstem als auch unter dem zweiten Buchstabe abgelegt. 
Ein Tupel wie \texttt{W:S} ist also sowohl unter \texttt{W} als auch unter \texttt{S} abgelegt.
Ein Tiefensuche Algorithmus, der modifiziert wurde, um weniger \glqq gierig\grqq{} zu agieren, schlägt nun die Tupel in der Tabelle nach und markiert die besuchten Tupel.
%Bei dem Rücksetzverfahren der Tiefensuche wird die Markierung der Tupel, die nicht zum Zyklus beitragen entfernt.
Das Ergebnis ist eine gemessene, lineare Laufzeit.\footnote{Es wurden Crib-Längen bis 26 betrachtet, um zu garantieren, dass die lineare Laufzeit bei einer maximalen Länge von 13 Buchstaben gegeben ist.} 
Das Menü wird als Vektor von \glqq CribCipherTuple\grqq{} in einer Struktur mit der Länge abgelegt.
Da es erforderlich ist, eine eindeutige \glqq Route\grqq{} durch das Menü anzugeben, werden Tupel-Kombination bei den betroffenen Knoten als Ausleger angegeben. 
Die Bombe kann somit im Falle der \glqq Scramblersatz-Knappheit\grqq{} entscheiden, ob sie diesen (nicht notwendigen) Bestandteil des Menüs aufnehmen möchte.

\subsection{Scrambler}\label{subsec:impl_scrambler}
Wie in~\cref{subsec:vorbereitungen} erklärt, vernachlässigt die Bombe den Übertragzeitpunkt der Walzen.
%Enigma Software gegeben erwähnen?

\subsection{Terminal}\label{subsec:impl_terminal}
\subsection{In und Outs}\label{subsec:impl_in_und_outs}
\subsection{Diagonalbrett}\label{subsec:impl_diagonal_board}
\subsection{Commons}\label{subsec:impl_commons}
\subsection{Test-Register}\label{subsec:impl_test-register}

\chapter{Geschwindigkeit der Software-Nachbildung im Kontext}