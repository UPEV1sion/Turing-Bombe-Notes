\chapter{Geschwindigkeit der Software-Nachbildung im Kontext}\label{ch:speed}

\section{Koinzidenzindex}\label{sec:koinzidenzindex}
Hatte die Bombe gestoppt, machten sich die Mitarbeiter in Bletchley Park daran, den Stopp zu "bewerten". 
Genauer wird überprüft, ob die Enigma-Konfiguration, signalisiert durch den Stopp, zu einem sinnvollen Ergebnis führt.
%In Bletchley-Park wurde diese Bewertung durch einen oder mehrere Linguistikexperten durchgeführt.
Zunächst wird ein Maß benötigt, mit welchem sich teilweise/vollständig dechiffrierte Texte "bewerten" lassen.
Ein geeignetes Maß für die Annäherung eines Textes an authentisches Deutsch ist der Koinzidenzindex:

\fmbox{\textbf{Koinzidenzindex}\[IC = \frac{\sum_{i=A}^{Z}f_i(f_i-1)}{N(N-1)}\]}
	
	
Sei $f_i$ die Häufigkeit des Buchstabens in Abhängigkeit von $i$ und $N$ die Gesamtanzahl der Buchstaben.
%Wir Summieren also die Anzahl der Buchstabenpaare auf, und teilen diese durch die Anzahl der allgemein möglichen Buchstabenpaare.
Der Koinzidenzindex ist ein statistisches Maß zur Analyse der Häufigkeitsverteilung der Buchstaben.
Für einen Text, bestehend aus zufälligen Buchstaben, beträgt der $IC \approx 0.038$, wobei er für Deutsch $\approx 0.078$ beträgt.
Ermittelt die Nachbildung einen Stopp, so wird der gesamte Geheimtext dechiffriert und mit dem Koinzidenzindex bewertet.
Gemessen wird der "Abstand" zum Referenzwert 0.078.
Texte, bei denen der Abstand des Koninzidenzindexes am geringsten ausfällt, werden am höchsten bewertet.


\section{Geschwindigkeit im Kontext der "echten" Bombe}\label{sec:speed}
Um die Geschwindigkeit der Software-Implementierung in einen Kontext zu setzen, muss die Entschlüsselungsprozedur genau analysiert werden.
In Bletchley Park waren zu Kriegsende mehr als 210 Bomben im Einsatz.
Der Bombe ist es möglich, eine Walzenlage innerhalb von etwa 20 Minuten abzuarbeiten.
Da über 210 Bomben zu Verfügung standen, war es möglich alle 60 Walzenlagen in dieser Zeit abzuarbeiten.
%Da das bewerten eine mühselige Handarbeit war, konnte die letztendliche Entschlüsselung des Tagesschlüssels 2-4 Stunden andauern.
Die Bestimmung des Menüs, Konfiguration der Bombe und die Bewertung der Walzenlagen in Bletchley Park war Handarbeit.
Deshalb variierte die Arbeitszeit bis zur Brechung des Schlüssels im Mittel zwischen 2-4 Stunden.
Die Bestimmung des Cribs beruhte einzig und allein auf der Expertise der Linguistischen Experten. 
Aus diesem Grund wurden manche Schlüssel gar nicht gebrochen.
Leider sind mögliche Heuristiken bei der Überprüfung der Stopps nicht überliefert.

\newpage
\begin{figure}[htbp]
	\centering
	\caption{Geschätzte Anzahl der Stops pro Walzenlage~\autocite{enwiki:bombe}}
	\label{fig:num-stops}
	\begin{tabular}{|c||c|c|c|c|c|c|c|c|c|}
		\hline
		& \multicolumn{9}{c|}{Buchstaben im Menü} \\
%		\hline
		Zyklen & 8 & 9 & 10 & 11 & 12 & 13 & 14 & 15 & 16 \\
		\hhline{|=||=|=|=|=|=|=|=|=|=|}
		3 & 2.2  & 1.1  & 0.42  & 0.14  & 0.04  & <0.01  & <0.01  & <0.01  & <0.01 \\
		\hline
		2 & 58  & 28  & 11  & 3.8  & 1.2  & 0.30  & 0.06  & <0.01  & <0.01  \\
		\hline
		1 & 1500  & 720  & 280  & 100  & 31  & 7.7  & 1.6  & 0.28  & 0.04  \\
		\hline
		0 & 40000  & 19000  & 7300  & 2700  & 820  & 200  & 43  & 7.3  & 1.0  \\
		\hline
	\end{tabular}
\end{figure}


Wie in~\cref{fig:num-stops} zu sehen, hängen die Anzahl der Stopps stark von der "Güte" des Menüs ab.
Die Software-Implementation benötigt für Erstellen eines Menüs der Länge zwölf, die Abarbeitung aller Walzenlagen und das "bewerten" der Stopps im Mittel
%TODO checking machine time
$150\si{\ms}$.

\[
\frac{3\si{\hour} \cdot 60 \cdot 60 \cdot 1000}{150\si{\ms}} \approx 72000
\]

Ein moderner Computer\footnote{Systemspezifikationen im Anhang enthalten:~\cref{fig:sys-spec}.} ist somit um Faktor 72000 schneller, als der kombinierte Kriegsaufwand der knapp 9000 Bletchley Park Mitarbeiter.
Es sei angemerkt, dass die Bewertung der Stopps durch Linguistische Experten hochwertiger ist, als der Koinzidenzindex.
