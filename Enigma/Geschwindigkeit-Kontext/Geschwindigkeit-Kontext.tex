\chapter{Geschwindigkeit der Software-Nachbildung im Kontext}\label{ch:speed}

Um die Geschwindigkeit der Software-Implementierung in einen Kontext zu setzen, muss die Entschlüsselungsprozedur genau analysiert werden.
In Bletchley Park waren zu Kriegsende mehr als 210 Bomben im Einsatz.
Durch eine erhöhte Drehzahl der Scramblern konnte eine Walzenlage innerhalb von etwa sechs abgearbeitet werden.
Angenommen das Crib wurde richtig gewählt, konnten somit alle 60 Walzenlagen innerhalb von sechs Minuten abgearbeitet werden.
Da die Bestimmung des Menüs, Konfiguration der Bombe und die Bewertung der Walzenlagen in Bletchley Park Handarbeit war, konnte die Arbeitszeit für diese Arbeitsgänge variieren.

\begin{figure}[htbp]
	\centering
	\caption{Geschätzte Anzahl der Stops pro Walzenlage~\autocite{enwiki:bombe}}
	\label{fig:num-stops}
	\begin{tabular}{|c|c|c|c|c|c|c|c|c|c|}
		\hline
		& \multicolumn{9}{c|}{Buchstaben im Menü} \\
		\hline
		Zyklen & 8 & 9 & 10 & 11 & 12 & 13 & 14 & 15 & 16 \\
		\hline
		3 & 2.2  & 1.1  & 0.42  & 0.14  & 0.04  & <0.01  & <0.01  & <0.01  & <0.01 \\
		\hline
		2 & 58  & 28  & 11  & 3.8  & 1.2  & 0.30  & 0.06  & <0.01  & <0.01  \\
		\hline
		1 & 1500  & 720  & 280  & 100  & 31  & 7.7  & 1.6  & 0.28  & 0.04  \\
		\hline
		0 & 40000  & 19000  & 7300  & 2700  & 820  & 200  & 43  & 7.3  & 1.0  \\
		\hline
	\end{tabular}
\end{figure}


Wie in~\cref{fig:num-stops} zu sehen, hängen die Anzahl der Stopps stark von der "Güte" des Menüs ab.
Die Software-Implementation benötigt für erstellen des Menüs der Länge zwölf, die Abarbeitung aller Walzenlagen und das "bewerten" der Stopps im Mittel 
%TODO checking machine time
$8.5\si{\sec}$.

\[
\]
